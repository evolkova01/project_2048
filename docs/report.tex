%%% Для сборки выполнить 2 раза команду: pdflatex <имя файла>

\documentclass[a4paper,12pt]{article}

\usepackage{ucs}
\usepackage[utf8x]{inputenc}
\usepackage[russian]{babel}
%\usepackage{cmlgc}
\usepackage{graphicx}
\usepackage{hyperref}
\usepackage{listings}
\usepackage{xcolor}
%\usepackage{courier}

\makeatletter
\renewcommand\@biblabel[1]{#1.}
\makeatother

\newcommand{\myrule}[1]{\rule{#1}{0.4pt}}
\newcommand{\sign}[2][~]{{\small\myrule{#2}\\[-0.7em]\makebox[#2]{\it #1}}}

% Поля
\usepackage[top=20mm, left=30mm, right=10mm, bottom=20mm, nohead]{geometry}
\usepackage{indentfirst}

% Межстрочный интервал
\renewcommand{\baselinestretch}{1.50}


\begin{document}

%%%%%%%%%%%%%%%%%%%%%%%%%%%%%%%
%%%                         %%%
%%% Начало титульного листа %%%

\thispagestyle{empty}
\begin{center}


\renewcommand{\baselinestretch}{1}
{\large
{\sc Петрозаводский государственный университет\\
Институт математики и информационных технологий\\
	Кафедра Информатики и Математического Обеспечения
}
}

\end{center}


\begin{center}
%%%%%%%%%%%%%%%%%%%%%%%%%
%
% Раскомментируйте (уберите знак процента в начале строки)
% для одной из строк типа направления  - бакалавриат/
% магистратура и для одной из
% строк Вашего направление подготовки
%
% Направление подготовки бакалавриата \\
% 01.03.02 Прикладная математика и информатика \\
% 09.03.02 - Информационные системы и технологии \\
09.03.04 - Программная инженерия \\
% Направление подготовки магистратуры \\
% 01.04.02 - Прикладная математика и информатика \\
% 09.04.02 - Информационные системы и  технологии \\
%
% 
%%%%%%%%%%%%%%%%%%%%%%%%%
	% \textcolor{red}{<Ваши тип и направление подготовки>} 
\end{center}

\vfill

\begin{center}
{\normalsize Отчет о проектной работе по курсу <<Основы информатики и программирования>>} \\

\medskip

%%% Название работы %%%
	{\Large \sc Разработка игрового приложения <<2048>>} \\
	% (промежуточный)
\end{center}

\medskip

\begin{flushright}
\parbox{11cm}{%
\renewcommand{\baselinestretch}{1.2}
\normalsize
	Выполнила:\\
%%% ФИО студента %%%
студентка 1 курса группы 22107
\begin{flushright}
	Е. Ф. Волкова \sign[подпись]{4cm}
\end{flushright}

Руководитель:\\
А. В. Бородин, старший преподаватель \\
% \begin{flushright}
% \sign[подпись]{4cm}
% \end{flushright}

}
\end{flushright}

\vfill

\begin{center}
\large
    Петрозаводск --- 2021
\end{center}

%%% Конец титульного листа  %%%
%%%                         %%%
%%%%%%%%%%%%%%%%%%%%%%%%%%%%%%%

%%%%%%%%%%%%%%%%%%%%%%%%%%%%%%%%
%%%                          %%%
%%% Содержание               %%%

\newpage

\hypersetup{hidelinks}
\tableofcontents

\newpage
\section*{Введение}
\addcontentsline{toc}{section}{Введение}


Цель проекта: разработать игровое приложение, которое реализует игру «2048» на языке С++ и QML. \\

Задачи проекта: 
\begin{enumerate}
    \item Изучить различные вариации игры <<2048>> и на их примере разработать требования к собственному приложению.
    \item Разработать графический интерфейс пользователя.
    \item Реализовать приложение с использованием разработанных модулей и QtQuick.
    \item Получить навыки по составлению документации, описывающей работу программы. 
\end{enumerate}

Все, кто имеет дело с компьютером, так или иначе сталкивались с компьютерными играми, и подавляющее большинство может сходу назвать несколько игр, которые им особенно понравились. Те, кто уже совсем наигрался, почти наигрался или еще не наигрался, но в процессе общения с компьютером уже начал совмещать игры с чем-нибудь более полезным, возможно, хотели бы придумать какие-нибудь свои, не похожие ни на какие другие игры. Многое захватывает в таком творчестве. В данном проекте речь пойдет о создании игровой программы «2048», которая и будет являться объектом моей работы.

%%%                          %%%
%%%%%%%%%%%%%%%%%%%%%%%%%%%%%%%%

\newpage

%%%%%%%%%%%%%%%%%%%%%%%%%%%%%%%%
%%%                          %%%
%%% Требования к приложению  %%%

\section{Требования к приложению}
\begin{itemize}
    \item Логика исходя из правил головоломки. 
    \item Меню, в котором можно просматривать данные о игре: счёт, лучший результат и непосредственно переход к самой игре.
    \item Выбор размера игрового поля.
    \item Приятный интерфейс.
    \item Вывод результатов игры на экран: победа (в случае достижения ячейки со значением 2048) или же поражение (в случае отсутствия хода).
    
\end{itemize}

%%%                          %%%
%%%%%%%%%%%%%%%%%%%%%%%%%%%%%%%%

\newpage

%%%%%%%%%%%%%%%%%%%%%%%%%%%%%%%%%
%%%                           %%%
%%% Проектирование приложения %%%
\section{Проектирование приложения}
Программа будет состоять из следующих основных функциональных частей:
\begin{itemize}
    \item Модуль, отвечающий за создание новой игры или завершение начавшейся.
    \item Модуль, отвечающий за размерность поля и генерацию новых ячеек.
    \item Модуль, реализующий движение ячеек и возможность слияния.
    \item Модуль, отвечающий за изменение цвета и значения при объединении ячеек.
    \item Модуль, отвечающий за подсчет очков.
\end{itemize}

%%%                          %%%
%%%%%%%%%%%%%%%%%%%%%%%%%%%%%%%%

\newpage

%%%%%%%%%%%%%%%%%%%%%%%%%%%%%%%%%
%%%                           %%%
%%% Реализация приложения     %%%
\section{Релизация приложения}
Для реализации игры были использованы такие языки программирования, как <<C++>> и <<QML>>. 
\begin{itemize}
    \item Количество <<C++>> файлов: 3.
    \begin{enumerate}
        \item board.cpp --- модуль, отвечающий за игровое поле.
        \item brick.cpp --- модуль, отвечающий за ячейки.
        \item main.cpp --- главный модуль для работы с функциями на языке «С++».
    \end{enumerate}
    \item Количество <<QML>> файлов: 1.
    \begin{enumerate}
        \item main.qml --- главный модуль графического интерфейса.
    \end{enumerate}

\end{itemize}

\begin{figure}[ht!]
\centering
\includegraphics[width=.6\textwidth]{filename}
\caption{Игра <<2048>>}
\label{fig:filename}
\end{figure}

%%%                          %%%
%%%%%%%%%%%%%%%%%%%%%%%%%%%%%%%%

\newpage

%%%%%%%%%%%%%%%%%%%%%%%%%%%%%%%%%
%%%                           %%%
%%% Заключение                %%%

\section*{Заключение}
\addcontentsline{toc}{section}{Заключение}

Так, мною была разработана игра <<2048>> - увлекательное приложение, которое поможет вам скоротать время и развить свою логику. Был получен  опыт работы не только с языком <<C++>>, но и с <<QtQuick>>. Написание программы способствовало закреплению теоретического материала на практике.\\

Игровое приложение <<2048>> является логически завершенной игрой. Также возможны изменения и добавления некоторых моментов в геймплей, которые можно реализовать в дальнейшем.


\end{document}
