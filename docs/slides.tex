\documentclass[10pt]{beamer}
\usepackage[T1,T2A]{fontenc}
\usepackage[utf8]{inputenc}
\usepackage{hyperref}
\hypersetup{unicode=true}
\usepackage{fontawesome}
\usepackage{graphicx}
\usepackage[english,russian]{babel}

\usepackage[T1]{fontenc}
\usepackage{fontawesome}
\usepackage{PTSans} 
\mode<presentation>
{
  \usetheme[progressbar=foot,numbering=fraction,background=light]{metropolis} 
  \usecolortheme{default}
  \usefonttheme{default}
  \setbeamertemplate{navigation symbols}{}
  \setbeamertemplate{caption}[numbered]
} 

\let\textttorig\texttt
\renewcommand<>{\texttt}[1]{%
  \only#2{\textttorig{#1}}%
}

\usepackage{minted}

\usepackage{xcolor}
\definecolor{codecolor}{HTML}{FFC300}

\usepackage{tcolorbox}
\tcbuselibrary{most,listingsutf8,minted}

\tcbset{tcbox width=auto,left=1mm,top=1mm,bottom=1mm,
right=1mm,boxsep=1mm,middle=1pt}

\newtcblisting{myr}[1]{colback=codecolor!5,colframe=codecolor!80!black,listing only, 
minted options={numbers=left, style=tcblatex,fontsize=\tiny,breaklines,autogobble,linenos,numbersep=3mm},
left=5mm,enhanced,
title=#1, fonttitle=\bfseries,
listing engine=minted,minted language=r}

\definecolor{mygreen}{HTML}{37980D}
\definecolor{myblue}{HTML}{0D089F}
\definecolor{myred}{HTML}{98290D}

\usepackage{listings}

\lstdefinelanguage{XML}
{
  morestring=[b]",
  morecomment=[s]{<!--}{-->},
  morestring=[s]{>}{<},
  morekeywords={ref,xmlns,version,type,canonicalRef,metr,real,target}
}

\lstdefinestyle{myxml}{
language=XML,
showspaces=false,
showtabs=false,
basicstyle=\ttfamily,
columns=fullflexible,
breaklines=true,
showstringspaces=false,
breakatwhitespace=true,
escapeinside={(*@}{@*)},
basicstyle=\color{mygreen}\ttfamily,
stringstyle=\color{myred},
commentstyle=\color{myblue}\upshape,
keywordstyle=\color{myblue}\bfseries,
}


% ------------------------------------------------------------------------------
% The Document
% ------------------------------------------------------------------------------
\title{Разработка игрового приложения <<2048>>}
\subtitle{Отчет о проектной работе по курсу <<Основы информатики и программирования>>}
\author{Екатерина Федоровна Волкова}
\date{11 июня 2021}

\begin{document}

\maketitle

\begin{frame}{Введение}
    Все, кто имеет дело с компьютером, так или иначе сталкивались с компьютерными
играми, и подавляющее большинство может сходу назвать несколько игр, которые им
особенно понравились. Те, кто уже совсем наигрался, почти наигрался или еще не наигрался, но в процессе общения с компьютером уже начал совмещать игры с чем-нибудь более полезным, возможно, хотели бы придумать какие-нибудь свои, не похожие ни на какие другие игры. Многое захватывает в таком творчестве. В данном проекте речь пойдет о создании игровой программы «2048», которая и будет являться объектом моей работы.
\end{frame}

\begin{frame}
  \frametitle{Цель и задачи проекта}
    \begin{block}{Цель проекта:}
        разработать игровое приложение, которое реализует игру «2048» на языке С++ и QML.
  \end{block}
  \begin{block}{Задачи проекта:}
  \begin{itemize}
    \item изучить различные вариации игры «2048» и на их примере разработать требования к собственному приложению;
    \item разработать графический интерфейс пользователя;
    \item реализовать приложение с использованием разработанных модулей и QtQuick;
    \item получить навыки по составлению документации, описывающей работу программы;
  \end{itemize}
  \end{block}
\end{frame}

\begin{frame}{План разработки игры}
    \begin{enumerate}
         \item Разработка модуля для движения ячеек и их объединения.
         \item Разработка модуля, отвечающего за создание новой игры(а также выбора размера игрового поля) или завершение начавшейся.
         \item Разработка модуля, реализующего генерацию новых ячеек. 
         \item Разработка модуля, характеризующего подсчет очков.
         \item Разработка модуля дизайна поля и ячеек.
    \end{enumerate}
\end{frame}

\begin{frame}{brick.cpp}
    \begin{itemize}
         \item Содержит модуль, отвечающий за действия с ячейками, например, такие как значение нашего кирпичика(setvalue()) и присвоение цвета (setcolors());
         \item А также получение индексов ячейки (setrow(); setcol()).
    \end{itemize}
\end{frame}

\begin{frame}{board.cpp}
    \begin{itemize}
        \item Состоит из модулей, отвечающих за движение ячеек (goUP(), goDown(), goLeft(), goRight()); генерацию новых (createNewBlock()) и их объединение moveBlocks(), (fusionBlocks());
        \item Также реализован следующий модуль: создание новой игры --- newGame(), конец раунда --- endGame(), победа в игре --- winGame(). 
        \item В дополнение ко всему был разработан модуль по подсчету очков (getscore(), getbestscore()) и установление размера поля (setdimension()).
    \end{itemize}
\end{frame}

\begin{frame}{main.qml}
    main.qml является главным модулем графического интерфейса, который содержит:
    \begin{enumerate}
        \item кнопки начала новой игры и выбора размера сетки.
        \item панель с текущим счетом и лучшим результатом.
        \item игровое поле вместе с ячейками.
    \end{enumerate}
\end{frame}

\begin{frame}{Реализация приложения}
        \begin{columns}[T,onlytextwidth]
                \begin{column}{0.44\textwidth}
                        \begin{itemize}
                                \item Таким образом, мною была разработана игра «2048» - увлекательное приложение, которое поможет вам скоротать время и развить свою логику. 
                                \item Для этого были использованы языки <<С++>> и <<QML>>.
                        \end{itemize}
                \end{column}
                \begin{column}{0.45\textwidth}
                        \includegraphics[width=\textwidth]{filename.JPG}
                \end{column}
        \end{columns}
\end{frame}

\begin{frame}
    \frametitle{Заключение}
    Игровое приложение «2048» является логически завершенной игрой. Также возможны
изменения и добавления некоторых моментов в геймплей, которые можно реализовать в
дальнейшем. 
    
\end{frame}

\begin{frame}[standout]
    Спасибо за внимание !
\end{frame}

\end{document}
